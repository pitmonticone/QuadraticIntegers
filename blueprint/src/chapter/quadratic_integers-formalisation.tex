\chapter{Ring of Integers of a Quadratic Field (Formalisation-Oriented)}

Let $d$ be an integer different from $0$ and $1$. We also assume that $d$ is squarefree.
We write $K$ for the $\Q$-algebra generated by $\sqrt{d}$: to be precise, we implement $K$ using \verb|QuadraticAlgebra ℚ d 0|.

\begin{lemma} \label{field}
  \lean{QuadraticInteger.field}
  \leanok
  For all rational $r$, we have $r^2 \neq d$, so $K$ is a field.
\end{lemma}
\begin{proof}
  Clear since we assume that $d$ is squarefree.
\end{proof}

\begin{lemma} \label{d_congr}
  We have that $d = \pm 1 \bmod 4$ or $d = 2 \bmod 4$.
  \lean{QuadraticInteger.d_congr}
  \leanok
\end{lemma}
\begin{proof}
  If $d = 0 \bmod 4$ than $d$ would not be squarefree.
\end{proof}

We now write $R$ for $\Z[\sqrt{d}]$: we have that $K$ is an $R$-algebra in the obvious way. Note that $R$ is implemented
as \verb|QuadraticAlgebra ℤ d 0|.

\begin{lemma} \label{easy_incl}
  We have that $\sqrt{d}$ is an integral element of $K$.
  \lean{QuadraticInteger.easy_incl}
  \leanok
\end{lemma}
\begin{proof}
  Clear since $\sqrt{d}$ is a root of $x^2-d$.
\end{proof}
Note that, in Lean, the above proposition technically says that that the image in $K$ of $\sqrt{d}$ \emph{as element of $R$} is integral.

\section{Trace and norm}
We fix in this section two rational numbers $a, b \in \Q$ and we write $z$ for
$a + b \sqrt{d} \in K$.

\begin{lemma} \label{rational_iff}
  \lean{QuadraticInteger.rational_iff}
  \leanok
  We have that $z \in \Q$ if and only if $b = 0$.
\end{lemma}
\begin{proof}
  Clear.
\end{proof}

\begin{lemma} \label{minpoly}
  If $b \neq 0$ then the minimal polynomial of $z$ over $\Q$ is
  \[
  X^2-2aX+(a^2-db^2)
  \]
  \lean{QuadraticInteger.minpoly}
  \leanok
\end{lemma}
\begin{proof}
  \uses{rational_iff}
  It's clear that $z$ is a root of $P$ and that $P \in \Q[X]$ is monic.

  Irreducibility follows by the fact that $P$ has a root that is irrational.
\end{proof}

\begin{lemma} \label{trace}
  \lean{QuadraticInteger.trace}
  \leanok
  We have that the trace of $z$ is $2a$.
\end{lemma}
\begin{proof}
  \uses{minpoly, field}
  If $b = 0$ then $z = a \in \Q$ and the trace is $2a$ since $[K : \Q] = 2$. Otherwise this is clear by Lemma \ref{minpoly}.
\end{proof}

\begin{lemma} \label{norm}
  \lean{QuadraticInteger.norm}
  \leanok
  We have that the norm of $z$ is $a^2-db^2$.
\end{lemma}
\begin{proof}
  \uses{minpoly, field}
  If $b = 0$ then $z = a \in \Q$ and the norm is $a^2$ since $[K : \Q] = 2$. Otherwise this is clear by Lemma \ref{minpoly}.
\end{proof}

\subsection{Integrality}
We now suppose that $z \in \mathcal{O}_K$.

\begin{lemma} \label{trace_int}
  \lean{QuadraticInteger.trace_int}
  \leanok
  We have that $2a \in \Z$
\end{lemma}
\begin{proof}
  \uses{trace}
  Since the trace of an algebraic integer is an integers, this follows by Lemma \ref{trace}.
\end{proof}

\begin{definition} \label{t_spec}
  \leanok
  \lean{QuadraticInteger.t_spec}
  \uses{trace_int}
  We write $t$ (for trace) to denote $2a$ as an integer. Mathematically we have $t = 2a$.
\end{definition}

\begin{lemma} \label{norm_int}
  \lean{QuadraticInteger.norm_int}
  \leanok
  We have that $a^2-db^2 \in \Z$
\end{lemma}
\begin{proof}
  \uses{norm}
  Since the norm of an algebraic integer is an integers, this follows by Lemma \ref{norm}.
\end{proof}

\begin{definition} \label{n_spec}
  \leanok
  \lean{QuadraticInteger.n_spec}
  \uses{norm_int}
  We write $n$ (for norm) to denote $a^2-db^2$ as an integer. Mathematically we have $n = a^2-db^2$.
\end{definition}

\begin{lemma} \label{four_n}
  We have that $4n = (2a)^2 - d(2b)^2$.
  \leanok
  \lean{QuadraticInteger.four_n}
\end{lemma}
\begin{proof}
  \uses{n_spec}
  Obvious.
\end{proof}

\begin{lemma} \label{squarefree_mul}
  \leanok
  \lean{squarefree_mul}
  Let $n$ be a squarefree integer and let $r$ be a rational such that $b r^2$ is an integer. Then $r$ is
  itself an integer.
\end{lemma}
\begin{proof}
  Easy.
\end{proof}

\begin{lemma} \label{two_b_int}
  \leanok
  \lean{QuadraticInteger.two_b_int}
  We have that $2b \in \Z$.
\end{lemma}
\begin{proof}
  \uses{four_n, squarefree_mul}
  By Lemma \ref{four_n}, $(2a)^2 - d(2b)^2$ is an integer and so, by Lemma \ref{trace_int},
  we know that $d(2b)^2 \in \Z$. Since $d$ is squarefree, we conclude that $2b \in \Z$ by Lemma \ref{squarefree_mul}.
\end{proof}

\begin{definition} \label{B₂_spec}
  \leanok
  \lean{QuadraticInteger.B₂_spec}
  \uses{two_b_int}
  We write $B_2$ to denote $2b$ as an integer. Mathematically we have $B_2 = 2b$.
\end{definition}

\begin{lemma} \label{b_int_of_a_int}
  \leanok
  \lean{QuadraticInteger.b_int_of_a_int}
  If $a \in \Z$ then $b \in \Z$.
\end{lemma}
\begin{proof}
  \uses{four_n, squarefree_mul}
  By Lemma \ref{four_n} and our assumption, both $(2a)^2$ and $(2a)^2 - d(2b)^2$ are integers divisible by $4$,
  so the same holds for $d(2b)^2$. In particular $db^2 \in \Z$ and $b \in \Z$ by Lemma \ref{squarefree_mul} since $d$ is squarefree.
\end{proof}

\begin{definition} \label{B_spec}
  \leanok
  \lean{QuadraticInteger.B_spec}
  \uses{b_int_of_a_int}
  If $a$ is an integer, we write $B$ to denote $b$ as an integer. Mathematically we have $B = b$.
\end{definition}

\begin{lemma} \label{a_not_int}
  \leanok
  \lean{QuadraticInteger.a_not_int}
  If $a \not\in \Z$ then $d = 1 \bmod{4}$.
\end{lemma}
\begin{proof}
  \uses{four_n, B₂_spec, t_spec}
  We have that $2a$, that is an integer, must be odd. By Lemmas \ref{four_n} and \ref{two_b_int}, we have
  $(2a)^2 = d(2b)^2 \bmod{4}$, so $2b$ must be odd and $d = 1 \bmod{4}$ as required.
\end{proof}

\section{The case \texorpdfstring{$d \neq 1 \bmod{4}$}{d not 1 mod 4}}

\begin{theorem} \label{d_2_or_3}
  \leanok
  \lean{QuadraticInteger.d_2_or_3}
  Assume that $d = 2 \bmod{4}$ or $d = 3 \bmod{4}$. Then
  \[
  \mathcal{O}_K = \Z[\sqrt(d)]
  \]
\end{theorem}
\begin{proof}
  \uses{easy_incl, a_not_int, d_congr, B_spec}
  By Lemma \ref{easy_incl} we know that $\Z[\sqrt(d)] \subseteq \mathcal{O}_K$. Let $z = a + b \sqrt{d} \in \mathcal{O}_K$, with
  $a, b \in \Q$. By Lemma \ref{a_not_int} we have that $a \in \Z$ (since by Lemma \ref{d_congr} we cannot have
  $d = 1 \bmod{4}$), and so by Lemma \ref{b_int_of_a_int} we have $b \in \Z$, so $z \in \Z[\sqrt{d}]$.
\end{proof}

\section{The case \texorpdfstring{$d = 1 \bmod{4}$}{d = 1 mod 4}}
We assume in this section that $d = 1 \bmod{4}$ and we write $e = \frac{d-1}{4}$.

\begin{lemma} \label{e_spec}
  \leanok
  \lean{QuadraticInteger.e_spec}
  We have that $e$ is an integer and $4e = d - 1$.
\end{lemma}
\begin{proof}
  Obvious.
\end{proof}

We write $S$ for the ring $\Z \left [ \frac{1+\sqrt{d}}{2} \right ]$, implemented as \verb|QuadraticAlgebra ℤ e 1|.

\begin{lemma} \label{algebra_R_S}
  \leanok
  \lean{QuadraticInteger.algebra_R_S}
  We have that
  \[
  \left(2 \left( \frac{1+\sqrt{d}}{2} \right) - 1 \right)^2 = d
  \]
  so that $S$ is an $R$-algebra.
\end{lemma}
\begin{proof}
  \uses{e_spec}
  Obvious by Lemma \ref{e_spec}.
\end{proof}

\begin{lemma} \label{algebra_S_K}
  \leanok
  \lean{QuadraticInteger.algebra_S_K}
  We have that
  \[
  \left( \frac{1+\sqrt{d}}{2} \right)^2 = \left( \frac{1+\sqrt{d}}{2} \right) + e
  \]
  so that $K$ is an $S$-algebra.
\end{lemma}
\begin{proof}
  \uses{e_spec}
  Obvious by Lemma \ref{e_spec}.
\end{proof}

\begin{lemma} \label{commutes_R_S_K}
  \leanok
  \lean{QuadraticInteger.commutes_R_S_K}
  The obvious diagram between $R$, $S$ and $K$ commutes.
\end{lemma}
\begin{proof}
  \uses{algebra_R_S, algebra_S_K}
  Clear.
\end{proof}

\begin{lemma} \label{easy_incl_d_1}
  \leanok
  \lean{QuadraticInteger.easy_incl_d_1}
  We have that $\frac{1+\sqrt{d}}{2} \in \mathcal{O}_K$.
\end{lemma}
\begin{proof}
  \uses{e_spec, algebra_S_K}
  Clear since $\frac{1+\sqrt{d}}{2}$ is a root of $X^2 - X - e \in \Z[X]$.
\end{proof}

\begin{lemma} \label{d_1_int}
  \leanok
  \lean{QuadraticInteger.d_1_int}
  Take $z = a + b \sqrt{d} \in \mathcal{O}_K$ with $a, b \in \Q$.
  If $a \in \Z$ then $t \in \Z\left[ \frac{1+\sqrt{d}}{2} \right]$.
\end{lemma}
\begin{proof}
  \uses{b_int_of_a_int}
  By Lemma \ref{b_int_of_a_int} we have that $b \in \Z$ and so $z \in \Z[\sqrt{d}] \subseteq \Z\left[ \frac{1+\sqrt{d}}{2} \right]$.
\end{proof}

\begin{theorem} \label{d_1}
  \leanok
  \lean{QuadraticInteger.d_1}
  We have
  \[
  \mathcal{O}_K = \Z\left[ \frac{1+\sqrt{d}}{2} \right]
  \]
\end{theorem}
\begin{proof}
  \uses{easy_incl_d_1, d_1_int, t_spec}
  By Lemma \ref{easy_incl_d_1} we know that $\Z\left[ \frac{1+\sqrt{d}}{2} \right] \subseteq \mathcal{O}_K$.
  Let $z = a + b \sqrt{d} \in \mathcal{O_K}$, with $a, b \in \Q$.
  \begin{itemize}
    \item If $ a \in \Z$ we conclude by Lemma \ref{d_1_int}.
    \item If $a \notin \Z$, let us consider
  \[
  z' = z - \frac{1+\sqrt{d}}{2} = a - \frac{1}{2} + \left( b - \frac{1}{2} \right) \sqrt{2} \in \mathcal{O}_K
  \]
  Since $2a \in \Z$ and $a \notin \Z$, we have that $a - \frac{1}{2} \in \Z$, so by Lemma \ref{d_1_int},
  we have that $z' \in \Z\left[ \frac{1+\sqrt{d}}{2} \right]$ and so $z \in \Z\left[ \frac{1+\sqrt{d}}{2} \right]$.
  \end{itemize}
\end{proof}
