\chapter{Ring of Integers of a Quadratic Field (Human-Oriented)}

Let $d$ be an integer different from $0$ and $1$. We also assume that $d$ is squarefree.
We write $K$ for the subring of $\C$ generated by $\sqrt{d}$. By our assumptions on $d$, we have that $K = \Q(\sqrt{d})$
is a field.

\begin{lemma} \label{d_congr}
  We have that $d \equiv \pm 1 \bmod 4$ or $d \equiv 2 \bmod 4$.
\end{lemma}
\begin{proof}
  If $d \equiv 0 \bmod 4$ than $d$ would not be squarefree.
\end{proof}

\begin{lemma} \label{easy_incl}
  We have that $\sqrt{d} \in \mathcal{O}_K$.
\end{lemma}
\begin{proof}
  Clear since $\sqrt{d}$ is a root of $x^2-d$.
\end{proof}

\begin{lemma} \label{easy_incl_d_1}
  If $d \equiv 1 \bmod{4}$ then $\frac{1+\sqrt{d}}{2} \in \mathcal{O}_K$.
\end{lemma}
\begin{proof}
  Write $d = 4a + 1$, with $a \in \Z$. Then $\frac{1+\sqrt{d}}{2}$ is a root of $x^2 - x - a \in \Z[x]$.
\end{proof}

Let $t \in K$, so $t = a + b \sqrt{d}$ for some $a, b \in \Q$. We assume that $t \notin \Q$, i.e. that
$b \neq 0$.

\begin{lemma} \label{minpoly}
  The minimal polynomial of $t$ over $\Q$ is
  \[
  P(x) = x^2-2at+(a^2-db^2)
  \]
\end{lemma}
\begin{proof}
  It's clear that $t$ is a root of $P$ and that $P \in \Q[x]$ is monic.

  Irreducibility follows by the fact that $P$ has a root that is not rational.
\end{proof}

\begin{lemma} \label{trace}
  We have that the trace of $t$ is $2a$.
\end{lemma}
\begin{proof}
  Clear by Lemma \ref{minpoly}.
\end{proof}

\begin{lemma} \label{norm}
  We have that the norm of $t$ is $a^2-db^2$.
\end{lemma}
\begin{proof}
  Clear by Lemma \ref{minpoly}.
\end{proof}

We suppose now that $t \in \mathcal{O}_K$.

\begin{lemma} \label{trace_int}
  We have that $2a \in \Z$
\end{lemma}
\begin{proof}
  Since the trace of an algebraic integer is an integers, this follows by Lemma \ref{trace}.
\end{proof}

\begin{lemma} \label{norm_int}
  We have that $a^2-db^2 \in \Z$
\end{lemma}
\begin{proof}
  Since the norm of an algebraic integer is an integers, this follows by Lemma \ref{norm}.
\end{proof}

\begin{lemma} \label{divisible}
We have that $(2a)^2 - d(2b)^2$. is an integer divisible by $4$.
\end{lemma}
\begin{proof}
  Clear since $(2a)^2 - d(2b)^2 = 4(a^2-db^2)$ and $a^2-db^2 \in \Z$ by Lemma \ref{norm_int}.
\end{proof}

\begin{lemma} \label{2b_int}
  We have that $2b \in \Z$.
\end{lemma}
\begin{proof}
  By Lemma \ref{divisible}, $(2a)^2 - d(2b)^2$ is an integer and so, by Lemma \ref{trace_int},
  we know that $d(2b)^2 \in \Z$. Since $d$ is squarefree, we conclude that $2b \in \Z$.
\end{proof}

\begin{lemma} \label{b_int_of_a_int}
  If $a \in \Z$ then $b \in \Z$.
\end{lemma}
\begin{proof}
  By Lemma \ref{divisible} and our assumption, both $(2a)^2$ and $(2a)^2 - d(2b)^2$ are integers divisible by $4$,
  so the same holds for $d(2b)^2$. In particular $db^2 \in \Z$ and $b \in \Z$ since $d$ is squarefree.
\end{proof}

\begin{lemma} \label{a_not_int}
  If $a \not\in \Z$ then $d \equiv 1 \bmod{4}$.
\end{lemma}
\begin{proof}
  We have that $2a$, that is an integer, must be odd. By Lemmas \ref{divisible} and \ref{2b_int}, we have
  $(2a)^2 \equiv d(2b)^2 \bmod{4}$, so $2b$ must be odd and $d \equiv 1 \bmod{4}$ as required.
\end{proof}

\begin{theorem} \label{d_2_or_3}
  Assume that $d \equiv 2 \bmod{4}$ or $d \equiv 3 \bmod{4}$. Then
  \[
  \mathcal{O}_K = \Z[\sqrt(d)]
  \]
\end{theorem}
\begin{proof}
  By Lemma \ref{easy_incl} we know that $\Z[\sqrt(d)] \subseteq \mathcal{O}_K$. Let $t = a + b \sqrt{d} \in \mathcal{O}_K$, with
  $a, b \in \Q$. By Lemma \ref{a_not_int} we have that $a \in \Z$ (since by Lemma \ref{d_congr} we cannot have
  $d \equiv 1 \bmod{4}$), and so by Lemma \ref{b_int_of_a_int} we have $b \in \Z$, so $t \in \Z[\sqrt{d}]$.
\end{proof}

\begin{lemma} \label{d_1_int}
  Assume that $d \equiv 1 \bmod{4}$ and take $t = a + b \sqrt{d} \in \mathcal{O}_K$ with $a, b \in \Q$.
  If $a \in \Z$ then $t \in \Z\left[ \frac{1+\sqrt{d}}{2} \right]$.
\end{lemma}
\begin{proof}
  By Lemma \ref{b_int_of_a_int} we have that $b \in \Z$ and so $t \in \Z[\sqrt{d}] \subseteq \Z\left[ \frac{1+\sqrt{d}}{2} \right]$.
\end{proof}

\begin{theorem} \label{d_1}
  Assume that $d \equiv 1 \bmod{4}$. Then
  \[
  \mathcal{O}_K = \Z\left[ \frac{1+\sqrt{d}}{2} \right]
  \]
\end{theorem}
\begin{proof}
  By Lemma \ref{easy_incl_d_1} we know that $\Z\left[ \frac{1+\sqrt{d}}{2} \right] \subseteq \mathcal{O}_K$.
  Let $t = a + b \sqrt{d} \in \mathcal{O_K}$, with $a, b \in \Q$.
  \begin{itemize}
    \item If $ a \in \Z$ we conclude by Lemma \ref{d_1_int}.
    \item If $a \notin \Z$, let us consider
  \[
  t' = t - \frac{1+\sqrt{d}}{2} = a - \frac{1}{2} + \left( b - \frac{1}{2} \right) \sqrt{2} \in \mathcal{O}_K
  \]
  Since $2a \in \Z$ and $a \notin \Z$, we have that $a - \frac{1}{2} \in \Z$, so by Lemma \ref{d_1_int},
  we have that $t' \in \Z\left[ \frac{1+\sqrt{d}}{2} \right]$ and so $t \in \Z\left[ \frac{1+\sqrt{d}}{2} \right]$.
  \end{itemize}
\end{proof}
